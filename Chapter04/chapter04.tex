\documentclass[../main.tex]{subfiles}

\pagestyle{main}
\renewcommand{\chaptermark}[1]{\markboth{\chaptername\ \thechapter: #1}{}}
\setcounter{chapter}{3}

\begin{document}




\chapter{Polynomials}
\begin{itemize}
    \item \marginnote{9/8:}\textbf{Real part} (of $a+bi\in\C$): The number $a$. \emph{Denoted by} $\bm{\re z}$.
    \item \textbf{Imaginary part} (of $a+bi\in\C$): The number $b$. \emph{Denoted by} $\bm{\im z}$.
    \item \textbf{Complex conjugate} (of $z\in\C$): The number $\re z-(\im z)i$. \emph{Denoted by} $\bm{\bar{z}}$.
    \item \textbf{Absolute value} (of $z\in\C$): The number $\sqrt{(\re z)^2+(\im z)^2}$. \emph{Denoted by} $\bm{|z|}$.
    \item $z=\bar{z}$ if and only if $z\in\R$.
    \item Properties of complex numbers.
    \begin{theorem}
        Suppose $w,z\in\C$. Then
        \begin{description}
            \item[sum of $\bm{z}$ and $\bm{\bar{z}}$]\hfill\\ $z+\bar{z}=2\re z$.
            \item[difference of $\bm{z}$ and $\bm{\bar{z}}$]\hfill\\ $z-\bar{z}=2(\im z)i$.
            \item[product of $\bm{z}$ and $\bm{\bar{z}}$]\hfill\\ $z\bar{z}=|z|^2$.
            \item[additivity and multiplicativity of the complex conjugate]\hfill\\ $\overline{w+z}=\bar{w}+\bar{z}$ and $\overline{wz}=\bar{w}\bar{z}$.
            \item[conjugate of conjugate]\hfill\\ $\overline{\bar{z}}=z$.
            \item[real and imaginary parts are bounded by $\bm{|z|}$]\hfill\\ $|\re z|\leq|z|$ and $|\im z|\leq|z|$.
            \item[absolute value of the complex conjugate]\hfill\\ $|\bar{z}|=|z|$.
            \item[multiplicativity of absolute value]\hfill\\ $|wz|=|w|\,|z|$.
            \item[Triangle Inequality]\hfill\\ $|w+z|\leq|w|+|z|$.
        \end{description}
    \end{theorem}
    \item If a polynomial is the zero function, then all coefficients are 0.
    \begin{itemize}
        \item It follows that the coefficients of a polynomial are uniquely determined.
    \end{itemize}
    \item \textbf{Division Algorithm} (for integers): If $p,s$ are nonnegative integers with $s\neq 0$, then there exist nonnegative integers $q,r$ such that $r<s$ and
    \begin{equation*}
        p = sq+r
    \end{equation*}
    \item Analogously,
    \begin{theorem}[Division Algorithm for Polynomials]
        Suppose that $p,s\in\pol{\F}$, with $s\neq 0$. Then there exist unique polynomials $q,r\in\pol{\F}$ such that
        \begin{equation*}
            p = sq+r
        \end{equation*}
        and $\deg r<\deg s$.
        \begin{proof}
            Let $n=\deg p$ and $m=\deg s$. We divide into two cases ($n<m$ and $n\geq m$). If $n<m$, then take $q=0$ and $r=p$.\par\smallskip
            On the other hand, if $n\geq P$, then let $T:\pol[n-m]{\F}\times\pol[m-1]{\F}\to\pol[n](\F)$ be defined by
            \begin{equation*}
                T(q,r) = sq+r
            \end{equation*}
            We can easily confirm that $T$ is a linear map.\par
            We now seek to prove that $\nul T=\{(0,0)\}$. Let $(q,r)\in\nul T$ be arbitrary. Then $sq+r=0$. It follows that all coefficients of the polynomial $sq+r$ are zero. Consequently, $q=0$ and $r=0$, as desired. Therefore, $\dim\nul T=0$. Additionally, Theorem \ref{trm:dimensionProduct} implies that
            \begin{equation*}
                \dim(\pol[n-m]{\F}\times\pol[m-1]{\F}) = (n-m+1)+(m-1+1) = n+1
            \end{equation*}
            It follows by the \hyperref[trm:fundamentalTheoremLinearMaps]{Fundamental Theorem of Linear Maps} that
            \begin{align*}
                \dim\range T &= \dim(\pol[n-m]{\F}\times\pol[m-1]{\F})-\dim\nul T\\
                &= n+1\\
                &= \dim\pol[n]{\F}
            \end{align*}
            Thus, by Exercise \ref{exr:subspaceSameDim}, $\range T=\pol[n]{\F}$. Therefore, since $p\in\pol[n]{\F}$, we know that there exists $q\in\pol[n-m]{\F}$ and $r\in\pol[m-1]{\F}$ such that $p=T(q,r)=sq+r$.\par
            Additionally, we know that $q,r$ are unique: If there exist $q',r'$ such that $T(q',r')=p$, then $T(q-q',r-r')=p-p=0$, implying since $\nul T=\{(0,0)\}$ that $q-q'=0$ and $r-r'=0$, i.e., that $q=q'$ and $r=r'$.
        \end{proof}
    \end{theorem}
    \item \textbf{Zero} (of $p\in\pol{\F}$): A number $\lambda\in\F$ such that $p(\lambda)=0$. \emph{Also known as} \textbf{root}.
    \item \textbf{Factor} (of $p\in\pol{\F}$): A polynomial $s\in\pol{\F}$ such that there exists a polynomial $q\in\pol{\F}$ satisfying $p=sq$.
    \item We now relate zeroes and factors.
    \begin{theorem}
        Suppose $p\in\pol{\F}$ and $\lambda\in\F$. Then $p(\lambda)=0$ if and only if there is a polynomial $q\in\pol{\F}$ such that
        \begin{equation*}
            p(z) = (z-\lambda)q(z)
        \end{equation*}
        for every $z\in\F$.
    \end{theorem}
    \item Putting bounds on the number of zeroes a polynomial can have.
    \begin{theorem}
        Suppose $p\in\pol{\F}$ is a polynomial with degree $m\geq 0$. Then $p$ has at most $m$ distinct zeros in $\F$.
    \end{theorem}
    \item We cannot prove the following without complex analysis, but we will state it, regardless.
    \begin{theorem}[Fundamental Theorem of Algebra]
        Every nonconstant polynomial with complex coefficients has a zero.
    \end{theorem}
    \item The following proceeds immediately from the Fundamental Theorem of Algebra.
    \begin{theorem}\label{trm:polFactorization}
        If $p\in\pol{\C}$ is a nonconstant polynomial, then $p$ has a unique factorization (except for the order of the factors) of the form
        \begin{equation*}
            p(z) = c(z-\lambda_1)\cdots(z-\lambda_m)
        \end{equation*}
        where $c,\lambda_1,\dots,\lambda_m\in\C$.
    \end{theorem}
    \item We now explore some of the differences between $\R$ and $\C$.
    \begin{theorem}
        Suppose $p\in\pol{\C}$ is a polynomial with real coefficients. If $\lambda\in\C$ is a zero of $p$, then so is $\bar{\lambda}$.
    \end{theorem}
    \begin{theorem}
        Suppose $b,c\in\R$. Then there is a polynomial factorization of the form
        \begin{equation*}
            x^2+bx+c = (x-\lambda_1)(x-\lambda_2)
        \end{equation*}
        with $\lambda_1,\lambda_2\in\R$ if and only if $b^2\geq 4c$.
    \end{theorem}
    \begin{theorem}
        Suppose $p\in\pol{\R}$ is a nonconstant polynomial. Then $p$ has a unique factorization (except for the order of factors) of the form
        \begin{equation*}
            p(x) = c(x-\lambda_1)\cdots(x-\lambda_m)(x^2+b_1x+c_1)\cdots(x^2+b_Mx+c_M)
        \end{equation*}
        where $c,\lambda_1,\dots,\lambda_m,b_1,\dots,b_M,c_1,\dots,c_M\in\R$, with $b_j^2<4c_j$ for each $j$.
    \end{theorem}
\end{itemize}




\end{document}