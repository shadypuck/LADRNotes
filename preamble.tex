\usepackage[margin=1in]{geometry}
\usepackage{fancyhdr}
\usepackage{csquotes}
\usepackage{marginnote}
\usepackage[style=apa]{biblatex}
\usepackage{xr}
\usepackage{enumitem}
\usepackage{scrextend}
\usepackage[bottom]{footmisc}
\usepackage{siunitx}
\usepackage{tikz}
\usepackage{float}
\usepackage{amsmath,amssymb,amsthm}
\usepackage{physics}
\usepackage{bm,scalerel,nicematrix}
\usepackage[hidelinks]{hyperref}

\fancypagestyle{main}{
    \fancyhf{}
    \fancyhead[L]{\leftmark}
    \fancyhead[R]{Abstract Linear Algebra Prep}
    \fancyfoot[R]{Labalme\ \thepage}
}
\fancypagestyle{plain}{
    \fancyhead{}
    \renewcommand{\headrulewidth}{0pt}
}

\MakeOuterQuote{"}

\reversemarginpar

\addbibresource{main.bib}
\DefineBibliographyStrings{english}{bibliography={References}}

\setitemize[3]{label={\scriptsize$\blacksquare$}}

\externaldocument{main}

\deffootnotemark{\textsuperscript{\textup{[}\thefootnotemark\textup{]}}}
\deffootnote[2.1em]{0em}{0em}{\textsuperscript{\thefootnote}}

\usetikzlibrary{calc}

\DeclareMathOperator{\spn}{span}
\DeclareMathOperator{\len}{len}
\DeclareMathOperator{\nul}{null}
\DeclareMathOperator{\range}{range}
\DeclareMathOperator{\re}{Re}
\DeclareMathOperator{\im}{Im}

\newtheorem{theorem}{Theorem}[chapter]
\newtheorem{lemma}[theorem]{Lemma}
\newtheorem*{lemma*}{Lemma}

\newcommand{\N}{\mathbb{N}}
\newcommand{\Z}{\mathbb{Z}}
\newcommand{\R}{\mathbb{R}}
\newcommand{\C}{\mathbb{C}}
\newcommand{\F}{\mathbb{F}}

\newcommand{\lin}[2]{\mathcal{L}(#1,#2)}
\newcommand{\ope}[1]{\mathcal{L}(#1)}
\newcommand{\mat}[1]{\mathcal{M}(#1)}
\newcommand{\pol}[2][]{\mathcal{P}_{#1}(#2)}
\newcommand{\inp}[2]{\left\langle{#1},{#2}\right\rangle}

\renewcommand{\qedsymbol}{$\blacksquare$}

\newcommand{\dq}[3][]{``#2"#1 \parencite[#3]{bib:Axler}.}

\renewcommand{\thesection}{\thechapter.\Alph{section}}

\newlength\shft
\shft=.15pt\relax
\newcommand{\fakebold}[1]{
    \ThisStyle{\ooalign{$\SavedStyle#1$\cr
    \kern-\shft$\SavedStyle#1$\cr
    \kern\shft$\SavedStyle#1$}}
}

\usepackage{subfiles}